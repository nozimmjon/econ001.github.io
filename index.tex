% Options for packages loaded elsewhere
\PassOptionsToPackage{unicode}{hyperref}
\PassOptionsToPackage{hyphens}{url}
\PassOptionsToPackage{dvipsnames,svgnames,x11names}{xcolor}
%
\documentclass[
  letterpaper,
  DIV=11,
  numbers=noendperiod,
  oneside]{scrreprt}

\usepackage{amsmath,amssymb}
\usepackage{iftex}
\ifPDFTeX
  \usepackage[T1]{fontenc}
  \usepackage[utf8]{inputenc}
  \usepackage{textcomp} % provide euro and other symbols
\else % if luatex or xetex
  \usepackage{unicode-math}
  \defaultfontfeatures{Scale=MatchLowercase}
  \defaultfontfeatures[\rmfamily]{Ligatures=TeX,Scale=1}
\fi
\usepackage{lmodern}
\ifPDFTeX\else  
    % xetex/luatex font selection
\fi
% Use upquote if available, for straight quotes in verbatim environments
\IfFileExists{upquote.sty}{\usepackage{upquote}}{}
\IfFileExists{microtype.sty}{% use microtype if available
  \usepackage[]{microtype}
  \UseMicrotypeSet[protrusion]{basicmath} % disable protrusion for tt fonts
}{}
\makeatletter
\@ifundefined{KOMAClassName}{% if non-KOMA class
  \IfFileExists{parskip.sty}{%
    \usepackage{parskip}
  }{% else
    \setlength{\parindent}{0pt}
    \setlength{\parskip}{6pt plus 2pt minus 1pt}}
}{% if KOMA class
  \KOMAoptions{parskip=half}}
\makeatother
\usepackage{xcolor}
\usepackage[left=1in,marginparwidth=2.0666666666667in,textwidth=4.1333333333333in,marginparsep=0.3in]{geometry}
\usepackage{soul}
\setlength{\emergencystretch}{3em} % prevent overfull lines
\setcounter{secnumdepth}{5}
% Make \paragraph and \subparagraph free-standing
\ifx\paragraph\undefined\else
  \let\oldparagraph\paragraph
  \renewcommand{\paragraph}[1]{\oldparagraph{#1}\mbox{}}
\fi
\ifx\subparagraph\undefined\else
  \let\oldsubparagraph\subparagraph
  \renewcommand{\subparagraph}[1]{\oldsubparagraph{#1}\mbox{}}
\fi


\providecommand{\tightlist}{%
  \setlength{\itemsep}{0pt}\setlength{\parskip}{0pt}}\usepackage{longtable,booktabs,array}
\usepackage{calc} % for calculating minipage widths
% Correct order of tables after \paragraph or \subparagraph
\usepackage{etoolbox}
\makeatletter
\patchcmd\longtable{\par}{\if@noskipsec\mbox{}\fi\par}{}{}
\makeatother
% Allow footnotes in longtable head/foot
\IfFileExists{footnotehyper.sty}{\usepackage{footnotehyper}}{\usepackage{footnote}}
\makesavenoteenv{longtable}
\usepackage{graphicx}
\makeatletter
\def\maxwidth{\ifdim\Gin@nat@width>\linewidth\linewidth\else\Gin@nat@width\fi}
\def\maxheight{\ifdim\Gin@nat@height>\textheight\textheight\else\Gin@nat@height\fi}
\makeatother
% Scale images if necessary, so that they will not overflow the page
% margins by default, and it is still possible to overwrite the defaults
% using explicit options in \includegraphics[width, height, ...]{}
\setkeys{Gin}{width=\maxwidth,height=\maxheight,keepaspectratio}
% Set default figure placement to htbp
\makeatletter
\def\fps@figure{htbp}
\makeatother
\newlength{\cslhangindent}
\setlength{\cslhangindent}{1.5em}
\newlength{\csllabelwidth}
\setlength{\csllabelwidth}{3em}
\newlength{\cslentryspacingunit} % times entry-spacing
\setlength{\cslentryspacingunit}{\parskip}
\newenvironment{CSLReferences}[2] % #1 hanging-ident, #2 entry spacing
 {% don't indent paragraphs
  \setlength{\parindent}{0pt}
  % turn on hanging indent if param 1 is 1
  \ifodd #1
  \let\oldpar\par
  \def\par{\hangindent=\cslhangindent\oldpar}
  \fi
  % set entry spacing
  \setlength{\parskip}{#2\cslentryspacingunit}
 }%
 {}
\usepackage{calc}
\newcommand{\CSLBlock}[1]{#1\hfill\break}
\newcommand{\CSLLeftMargin}[1]{\parbox[t]{\csllabelwidth}{#1}}
\newcommand{\CSLRightInline}[1]{\parbox[t]{\linewidth - \csllabelwidth}{#1}\break}
\newcommand{\CSLIndent}[1]{\hspace{\cslhangindent}#1}

\KOMAoption{captions}{tableheading}
\makeatletter
\makeatother
\makeatletter
\@ifpackageloaded{bookmark}{}{\usepackage{bookmark}}
\makeatother
\makeatletter
\@ifpackageloaded{caption}{}{\usepackage{caption}}
\AtBeginDocument{%
\ifdefined\contentsname
  \renewcommand*\contentsname{Table of contents}
\else
  \newcommand\contentsname{Table of contents}
\fi
\ifdefined\listfigurename
  \renewcommand*\listfigurename{List of Figures}
\else
  \newcommand\listfigurename{List of Figures}
\fi
\ifdefined\listtablename
  \renewcommand*\listtablename{List of Tables}
\else
  \newcommand\listtablename{List of Tables}
\fi
\ifdefined\figurename
  \renewcommand*\figurename{Figure}
\else
  \newcommand\figurename{Figure}
\fi
\ifdefined\tablename
  \renewcommand*\tablename{Table}
\else
  \newcommand\tablename{Table}
\fi
}
\@ifpackageloaded{float}{}{\usepackage{float}}
\floatstyle{ruled}
\@ifundefined{c@chapter}{\newfloat{codelisting}{h}{lop}}{\newfloat{codelisting}{h}{lop}[chapter]}
\floatname{codelisting}{Listing}
\newcommand*\listoflistings{\listof{codelisting}{List of Listings}}
\makeatother
\makeatletter
\@ifpackageloaded{caption}{}{\usepackage{caption}}
\@ifpackageloaded{subcaption}{}{\usepackage{subcaption}}
\makeatother
\makeatletter
\@ifpackageloaded{tcolorbox}{}{\usepackage[skins,breakable]{tcolorbox}}
\makeatother
\makeatletter
\@ifundefined{shadecolor}{\definecolor{shadecolor}{rgb}{.97, .97, .97}}
\makeatother
\makeatletter
\makeatother
\makeatletter
\@ifpackageloaded{sidenotes}{}{\usepackage{sidenotes}}
\@ifpackageloaded{marginnote}{}{\usepackage{marginnote}}
\makeatother
\makeatletter
\makeatother
\ifLuaTeX
  \usepackage{selnolig}  % disable illegal ligatures
\fi
\IfFileExists{bookmark.sty}{\usepackage{bookmark}}{\usepackage{hyperref}}
\IfFileExists{xurl.sty}{\usepackage{xurl}}{} % add URL line breaks if available
\urlstyle{same} % disable monospaced font for URLs
\hypersetup{
  pdftitle={Iqtisodchi va boshqalar uchun yon daftar},
  pdfauthor={Nozimjon Ortiqov},
  colorlinks=true,
  linkcolor={blue},
  filecolor={Maroon},
  citecolor={Blue},
  urlcolor={Blue},
  pdfcreator={LaTeX via pandoc}}

\title{Iqtisodchi va boshqalar uchun yon daftar}
\author{Nozimjon Ortiqov}
\date{2024-11-11}

\begin{document}
\maketitle
\ifdefined\Shaded\renewenvironment{Shaded}{\begin{tcolorbox}[borderline west={3pt}{0pt}{shadecolor}, interior hidden, breakable, enhanced, frame hidden, sharp corners, boxrule=0pt]}{\end{tcolorbox}}\fi

\renewcommand*\contentsname{Table of contents}
{
\hypersetup{linkcolor=}
\setcounter{tocdepth}{2}
\tableofcontents
}
\bookmarksetup{startatroot}

\hypertarget{soz-boshi}{%
\chapter*{SO'Z BOSHI}\label{soz-boshi}}
\addcontentsline{toc}{chapter}{SO'Z BOSHI}

\markboth{SO'Z BOSHI}{SO'Z BOSHI}

Iqtisodiyot qiyin fan, Statistika ham. Ushbu mini kitobda asosan
iqtisodiy statistikani talqin qilishda uchraydigan xatolar haqida so'z
yuritiladi.

Kitob iqtisodchilar va boshqa kasb egalari uchun mo'ljallangan bo'lib,
o'qish va tushunish uchun iqtisod bo'yicha bazaviy bilim talab
qilinmaydi.

\bookmarksetup{startatroot}

\hypertarget{kirish}{%
\chapter*{Kirish}\label{kirish}}
\addcontentsline{toc}{chapter}{Kirish}

\markboth{Kirish}{Kirish}

\bookmarksetup{startatroot}

\hypertarget{import-vs-yalpi-ichki-mahsulot-yaim}{%
\chapter*{Import vs Yalpi ichki mahsulot
(YaIM)}\label{import-vs-yalpi-ichki-mahsulot-yaim}}
\addcontentsline{toc}{chapter}{Import vs Yalpi ichki mahsulot (YaIM)}

\markboth{Import vs Yalpi ichki mahsulot (YaIM)}{Import vs Yalpi ichki
mahsulot (YaIM)}

\textbf{Yalpi ichki mahsulot (YaIM)} - \textbf{mamlakat hududida} ma'lum
davr (yil yoki chorak) mobaynida yaratilgan tovar va xizmatlarning
umumiy qiymati.

Ushbu ta'rifni bir muhim jihati bor:

\textbf{Mamlakat tashqarisida yaratilgan mahsulotlar (import) YaIM
tarkibiga kirmaydi}.

Ammo, juda ko'pchilik \emph{import YaIMni kamaytiradi deb o'ylaydi},
hattoki professional iqtisodchi va nufuzli jurnallar ham.

Ushbu xatolik milliy hisoblar tizimidagi ayniyatdan kelib chiqadi. Unga
ko'ra, YaIM xarajatlar usulida quyidagicha hisoblanadi:

\(Y = C + I + G+(X-M)\) .

\marginnote{\begin{footnotesize}

\emph{Bu yerda, Y - yalpi ichki mahsulot, C - iste'mol, I -
investitsiya, G - davlat xarajatlari, X - eksport, M - import.}

\end{footnotesize}}

Ushbu ayniyatda M - import komponenti minus sifatida ishtirok etishi
xato xulosaga olib keladi. Qolgan elementlar qo'shilyapti, import
ayrilyapti. Shundan kelib chiqib, ko'pchilik ``\emph{importni
kamaytirsak, YaIM ko'payadi}'', yoki ``\emph{import ko'paygani hisobiga
YaIM o'sishi sekinlashdi}'' deb xulosa qiladi.

Unda nega formulada import chegirilmoqda?

Sababi juda oddiy, \ul{\textbf{import YaIM tarkibida o'tirmaydi}}.

Iste'mol (investitsiya, davlat) xarajatlari tarkibida mamlakatda ishlab
chiqarilgan mahsulotlar bilan birga import ham o'tiradi.

Misol uchun, yangi Iphone telefon sotib oldingiz, bu iste'molingizning
bir qismi. Ammo, ichki, mahalliy mahsulot emas - O'zbekistonda ishlab
chiqarilmagan. Shuning uchun ushbu xarajatni YaIMni hisoblashda chegirib
tashlash kerak.

Savol tug'ilishi mumkin, nima uchun birdaniga iste'mol, investitsiya
yoki davlat xarajatlaridan importni chegirib, yuqoridagi tenglamadan
import komponenti olib tashlanmaydi va quyidagicha ifoda qilinmaydi?

\(Y = C + I + G+X\)

Sababi, iste'mol, investitsiya va davlat xarajatlari umumiy (aggregated)
ko'rsatkich va ushbu xarajatlarning kelib chiqish manbasi bo'yicha
alohida statistika yuritilmaydi. Sizdan yil davomidagi iste'molingizning
qanchasi import mahsulotlar deb so'rashsa, aniq ayta olmaysiz yoki
buning hisobini yuritmaysiz.

Shuning uchun statistikasi yuritiladigan umumiy iste'mol
(investitsiya)dan statistikasi bor bo'lgan umumiy import komponenti olib
tashlanadi.

Yana ham aniqroq tushunish uchun yuqoridagi YaIMni hisoblash formulasini
quyidagicha ifoda qilish mumkin:

\(Y = (C~d~ + C~f~ ) + (I~d~ + I~f~) + (G~d~ + G~f~) +(X- [C~f~ +I~f~ + G~f~])\)

Bunda Iste'mol (Investitsiya, Davlat) xarajatlari ichki va tashqi
komponentlarga ajratilgan.

Mazkur formula yordamida bir misol ko'raylik.

Tasavvur qilaylik, davlatda Iste'mol 100 birlikka, Investitsiya, Davlat
xarajatlari, Eksport va Import esa 0 bo'lsin. Shunda YaIM 100 ga teng
bo'ladi:

\(Y = (100+0) + 0 + 0 + (0-0)\)

Endi, ushbu gipotetik davlat tashqi bozorga ochilib, 100 birlik import
qilsin va uni iste'mol qilsin. U holda tenglikni quyidagicha yozish
mumkin:

\(Y = (100+100) + 0 + 0 + (0-100)\)

Ushbu misol soddalashtirilsa, YaIM 100 ga teng bo'ladi. Bunda iste'mol
200, ammo YaIM hajmi o'zgarmasdan qoladi.

Mazkur misoldan ko'rish mumkinki, \textbf{import YaIM hajmiga ta'sir
ko'rsatadi} va faqat bir xarajatni ikki marta hisobga olmaslik uchun
korrektirovka qilinadi.

Demak,

\begin{enumerate}
\def\labelenumi{\arabic{enumi}.}
\tightlist
\item
  YaIM mamlakat hududida yaratilgan tovar va xizmatlarni o'z ichiga
  oladi.
\item
  Import uchun sarflangan xarajatlarni YaIMni hisoblashda chegirib
  tashlash kerak.
\item
  Import to'g'ridan-to'g'ri YaIMni kamaytirmaydi.
\end{enumerate}

Endi, kimdir aytishi mumkin, import ko'paysa milliy mahsulotlarni sotib
olish kamayadi va bu YaIMni kamaytiradi deb. \textbf{Bu bizni yana bir
xatoga yetaklaydi - ayniyatdan (formula) sabab-oqibatni keltirib
chiqarish}. Bu haqida keyingi qismda to'xtalamiz.

\bookmarksetup{startatroot}

\hypertarget{ayniyatdan-xulosa-chiqarish-kerak-emas}{%
\chapter*{Ayniyatdan xulosa chiqarish kerak
emas}\label{ayniyatdan-xulosa-chiqarish-kerak-emas}}
\addcontentsline{toc}{chapter}{Ayniyatdan xulosa chiqarish kerak emas}

\markboth{Ayniyatdan xulosa chiqarish kerak emas}{Ayniyatdan xulosa
chiqarish kerak emas}

Ayniyat nima? Ayniyat - bu tarif bo'yicha doim to'g'ri degani. Iqtisodiy
muhokamalarda eng ko'p uchraydigan xatolardan biri ayniyat orqali
sabab-oqibat bo'yicha xulosa qilish.

Ingliz tilida ``Never reason from an accounting identities'' deb
aytishadi. Twitterda (X) ``Accounting identities'' deb qidirib ko'ring,
juda ko'p muhokamalar kelib chiqadi.

\begin{marginfigure}

{\centering \includegraphics{GFxc02uWcAASvTq.jpeg}

}

\end{marginfigure}

YaIMni hisoblashning quyida keltirilgan formulasi shunday ayniyatga
misol bo'la oladi:

\(Y = C + I + G+(X-M)\) .

\marginnote{\begin{footnotesize}

Ayniyatlarga misollar:\\
1. Pulning miqdoriy nazariyasi: \(M*V = P*T\) .\\
\emph{Bunda M - pul massasi (M2), V - pul aylanish tezligi, P-narxlar,
T-transaksiyalar soni.}

\end{footnotesize}}

Ushbu tenglikni asos qilib olib ko'pincha ``davlat xarajatlarini
ko'paytirish YaIMni oshiradi'' deyiladi. Yoki, G'ni oshirish YaIMni
ko'paytirmaydi, sababi agar G yuqori bo'lsa investitsiyani (I) siqib
chiqaradi (crowding out). Har ikkala fikr ham to'g'ri bo'lishi mumkin,
ammo buning uchun yuqoridagi formula asos emas.

Eng qizig'i yuqoridagi YaIM formulasini quyidagicha ham yozish mumkin:

\(Y = C + S + T\) .

Bu ayniyatdan ``soliqlarni ko'paytirish YaIMni oshiradi'' deb ham aytish
mumkin. Bu fikr deyarli ishlatilmaydi, lekin tepadagi bilan bir xil
asosga ega.

Yana bir misol, daromad - jamg'arma va xarajatlar yig'indisiga teng.

\(Daromad = Xarajat + Jamg'arma\)

Lekin hech qachon biz xarajatlarimizni ko'paytirsak daromadimiz
ko'payadi deb aytmaymiz, to'g'rimi? Buning barchasi nimani nima keltirib
chiqarishiga bog'liq. Misol uchun, xarajatni ko'paytirish ko'proq
ishlashga majbur qilar yoki kamroq jamg'arma qilarmiz.

Ayniyat - doim to'g'ri. Ayniyatdan to'g'ridan-to'g'ri sabab-oqibat
munosabatlarini o'rnatib bo'lmaydi, aks holda tavtologiyaga aylanib
qoladi.

Ayniyatlar nima uchun kerak unda? Ayniyatlar ``hisob-kitob balans
beradimi yoki yo'qmi?'' degan savolga javob topish uchun juda muhim,
sabab-oqibat munosabatlarini o'rnatish uchun emas.

Sabab-oqibat savollariga javob topish nazariylashtirish, xulq-atvor
munosabatlarini (behavioural equation) modellashtirish va ekonometrik
hisob-kitoblarni amalga oshirishni talab etadi.

\bookmarksetup{startatroot}

\hypertarget{inflyatsiya-va-narx-ozgarishi}{%
\chapter*{Inflyatsiya va narx
o'zgarishi}\label{inflyatsiya-va-narx-ozgarishi}}
\addcontentsline{toc}{chapter}{Inflyatsiya va narx o'zgarishi}

\markboth{Inflyatsiya va narx o'zgarishi}{Inflyatsiya va narx
o'zgarishi}

\bookmarksetup{startatroot}

\hypertarget{real-va-nominal-stavka}{%
\chapter*{Real va Nominal stavka}\label{real-va-nominal-stavka}}
\addcontentsline{toc}{chapter}{Real va Nominal stavka}

\markboth{Real va Nominal stavka}{Real va Nominal stavka}

\bookmarksetup{startatroot}

\hypertarget{foiz-va-foiz-bandi}{%
\chapter*{Foiz va Foiz bandi}\label{foiz-va-foiz-bandi}}
\addcontentsline{toc}{chapter}{Foiz va Foiz bandi}

\markboth{Foiz va Foiz bandi}{Foiz va Foiz bandi}

\bookmarksetup{startatroot}

\hypertarget{foydalanilgan-adabiyotlar}{%
\chapter*{Foydalanilgan adabiyotlar}\label{foydalanilgan-adabiyotlar}}
\addcontentsline{toc}{chapter}{Foydalanilgan adabiyotlar}

\markboth{Foydalanilgan adabiyotlar}{Foydalanilgan adabiyotlar}

\hypertarget{refs}{}
\begin{CSLReferences}{0}{0}
\end{CSLReferences}



\end{document}
